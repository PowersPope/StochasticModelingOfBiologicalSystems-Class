% This is a very basic LaTeX template for homework solutions in modeling courses.
% It demonstrates how to write math equations (inline and in separate lines), how to display a simple figure, how to create multiple sections, and how to add numberings to equations that you can then refer to in the text.
% Use and modify at your own risk.

% specify the general document type
% this should always come first
\documentclass[letter,12pt]{article}

% load packages for more math symbols
\usepackage{amsmath, amsthm, amsfonts, amscd, amsxtra, mathrsfs, amssymb, latexsym, stmaryrd, cancel, mathdots, wasysym}

% allow importing graphics
\usepackage{graphicx}

% specify input language and font encoding
\usepackage[english]{babel}
\usepackage[T1]{fontenc}
\usepackage[utf8]{inputenc}

% improve the way links appear
\usepackage[bookmarks,bookmarksnumbered,colorlinks=true,urlcolor=blue,linkcolor=blue,citecolor=blue]{hyperref}

% modify document margins
\usepackage{anysize}
\marginsize{2.5cm}{2.5cm}{1cm}{2cm}

% more sophisticated lists
\usepackage{enumerate}


% - - - - - - - - - - - - - TITLE PAGE - - - - - - - - - - - - %

% title
\title{\emph{Stochastic dynamical modeling in biology}\\\vspace{1em}- Notes Week 2 -}

% author
\author{Andrew Powers}

% disable date display
\date{}

\begin{document}

\maketitle % generate document title
\hrule % draw a horizontal line


% - - - - - - - - - - - - - START DOCUMENT - - - - - - - - - - - - %

\section*{Notes 04-11-23}
\subsection{Power Spectral Density}
Talking about power spectral density (PSD).

\begin{enumerate}[1.]
  \item Important information to remember:
  \begin{itemize}
    \item Lower bound of our time: $\frac{1}{M\tau}$ 
    \item Upper bound of our time: $\frac{1}{2\tau}$
  \end{itemize}
  \item When we havd some sort of periodicity. We see it in the autocovariance function and within the PSD.
  \item Real numbers have no phase. Complex numbers have a non-zero phase.
  \item A phase that is at zero, when comparing two other components, the two elements oscillate in synchrony.
    However, if they were not at zero then they would possibly oscilliate not together. Ex. Preditor and Prey.
  \item When applying a lot of normalization types like (Welch, Bartlett) method may produce a lot of noise within the 
    phase. However, if we can reproduce at the point of the peak in the modulus, and the phase are the same. Then this
    is a possible true synchrony we see.
\end{enumerate}

Questions
\begin

\subsection{Stochastic dynamical models}
\begin{enumerate}[1.]
  \item It won't tell us exactly what is going to happen. However, it will give us some idea about what will happen. i.e.,
    it will tell us what the expected or probability of something happening will be. 

  \item Stochastic Iterative Map
    \begin{equation}
      X_k \sim g(K, X_{k-1}, X_{k-2},...,X_{k-P})
    \end{equation}
    The random state of the system at time point k.
    \begin{itemize}
     \item If $X_k\in\mathbb{R}^n$ we call the SIM "numverical $n$-dimensional"
     \item If $P>1$ we call the SIM delayed and p is the order of the SIM
     \item If $P=1$ (immediately depends on the proceeding one) the SIM is called Markov
     \item If $g$ does not explicitly depend on $k$,the SIM is called autonomous or time-homogeneious (no external factors change the dependency of the $k$)
     \item Autonomous Markov SIM with a discrete state space are also known as discrete-time Markov chains
    \end{itemize}

    
    Example: 
    \begin{itemize}
      \item $N_k = rN_{k-1}$ Deterministic Model (Geometric growth)
      \item $N_k \sim \Pi(rN_{k-1})$ Stochastic Model -> Autonomous and markovian
    \end{itemize}

    Capital Pi ($\Pi$) to denote the poisson distribution. Only free parameter is the mean.

    New Example:
    \begin{itemize}
      \item If we suppose that r depends on time. We now have multiple r's. 
      \item $N_k \sim \Pi(r_kN_{k-1})\rightarrow$ Still Markovian. 
      \item However, it is no longer Autonomous as we have anther time related factor/outside factor that influences the growth.
    \end{itemize}


  
  
\end{enumerate}


\end{document}
